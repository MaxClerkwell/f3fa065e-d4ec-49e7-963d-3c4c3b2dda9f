\documentclass{dcbl/challenge}

\setdoctitle{Graphs and Heap}
\setdocauthor{Stephan Bökelmann}
\setdocemail{sboekelmann@ep1.rub.de}
\setdocinstitute{AG Physik der Hadronen und Kerne}


\begin{document}

Information is not always just a single piece of data, neither is it always a consecutive sequence of bytes.
In some cases, the data is structured or encoded in a way that several pieces make up the whole data-set.
Think of an application like facebook. 
Every account has some attributes, like name, birthday, etc., but also contains a list of friends.
Each element in the list of friends is structured the same way. 
The resulting data structure can be abstracted as a mathematical graph.
One possible implementation of a graph is called a linked list.
Each element of the linked list, contains a piece of usable data, as well as a pointer to the next neighbour, as well as a pointer to the previous neighbour.
Look at the following link for an \href{https://godbolt.org/#g:!((g:!((g:!((h:codeEditor,i:(filename:'1',fontScale:14,fontUsePx:'0',j:1,lang:___c,selection:(endColumn:2,endLineNumber:31,positionColumn:2,positionLineNumber:31,selectionStartColumn:2,selectionStartLineNumber:31,startColumn:2,startLineNumber:31),source:'%23include+%3Cstdio.h%3E%0A%0Atypedef+struct+node+node%3B%0A%0Astruct+node%7B%0A++++++++int+data%3B%0A++++++++node*+next%3B%0A%7D%3B%0A%0Aint+main()%7B%0A++++++++node+a+%3D+%7B%0A++++++++++++++++.data+%3D+2,%0A++++++++++++++++.next+%3D+NULL%0A++++++++%7D%3B%0A%0A++++++++node+b+%3D+%7B%0A++++++++++++++++.data+%3D+4,%0A++++++++++++++++.next+%3D+NULL%0A++++++++%7D%3B%0A%0A++++++++a.next+%3D+%26b%3B%0A%0A++++++++printf(%22+++Adresse+von+a:+%25p%5Cn%22,+%26a)%3B%0A++++++++printf(%22++++Inhalt+von+a:+%25d%5Cn%22,+a.data)%3B%0A++++++++printf(%22N%C3%A4chstes+Element:+%25p%5Cn%22,+a.next)%3B%0A++++++++printf(%22+++Adresse+von+b:+%25p%5Cn%22,+%26b)%3B%0A++++++++printf(%22++++Inhalt+von+b:+%25d%5Cn%22,+b.data)%3B%0A++++++++printf(%22N%C3%A4chstes+Element:+%25p%5Cn%22,+b.next)%3B%0A%0A++++++++return+0%3B%0A%7D'),l:'5',n:'0',o:'C+source+%231',t:'0')),k:51.65615141955836,l:'4',n:'0',o:'',s:0,t:'0'),(g:!((h:executor,i:(argsPanelShown:'1',compilationPanelShown:'0',compiler:cg131,compilerName:'',compilerOutShown:'0',execArgs:'',execStdin:'',fontScale:14,fontUsePx:'0',j:1,lang:___c,libs:!(),options:'',overrides:!(),runtimeTools:!(),source:1,stdinPanelShown:'1',wrap:'1'),l:'5',n:'0',o:'Executor+x86-64+gcc+13.1+(C,+Editor+%231)',t:'0')),header:(),k:48.34384858044164,l:'4',n:'0',o:'',s:0,t:'0')),l:'2',n:'0',o:'',t:'0')),version:4}{example of this}.

\section*{Exercises}
\begin{aufgabe}
    Write a program, that creates a graph of nodes. 
    Each node should contain a number as an integer. 
    Succeedingly write a function, that takes a pointer to a node and prints the value of the node. 
\end{aufgabe}

\begin{aufgabe}
    Now write a function, that contains a while loop.
    The function takes a pointer to a node and prints all numbers from the nodes, starting with the node it was given.
    At the end of the while loop, the loop variable gets set to the next node. 
    If there is no next node, the loop ends and the function returns. 
    You have now successfully travesed a graph! 
\end{aufgabe}

\begin{aufgabe}
    We can use graphs, to store data of arbitrary length (as long as it is smaller than our RAM).
    In order to do this, we need to be able to allocate memory dynamically.
    For this we will use the heap strategy. 
    We controlled the stack with the operators \texttt{\{} and \texttt{\}}.
    We control the heap with the operators \texttt{malloc()} and \texttt{free()}.
    \texttt{malloc()} takes the size in bytes as an argument and returns a pointer to the allocated memory.
    \texttt{free()} takes a pointer to the memory as an argument and frees the memory.
    Working with the heap can be dangerous.
    In order to use it correctly, more than just one day of training is required.
    Just take a look at the following program and try to understand what happens:
    \href{https://godbolt.org/#z:OYLghAFBqd5QCxAYwPYBMCmBRdBLAF1QCcAaPECAMzwBtMA7AQwFtMQByARg9KtQYEAysib0QXACx8BBAKoBnTAAUAHpwAMvAFYTStJg1AB9U8lJL6yAngGVG6AMKpaAVxYM9DgDJ4GmADl3ACNMYhAANgB2UgAHVAVCWwZnNw89eMSbAV9/IJZQ8OiLTCtshiECJmICVPdPLhKy5MrqglzAkLDImIUqmrr0xr62jvzCnoBKC1RXYmR2DgBSACYAZj9kNywAaiW1xz78VAA6BH3sJY0AQVWNhi3XXf3DgnRaPGCzi6vbm4IAJ6xTBYKg7PrEVzWHYMDCYPZRABCvx2qJ2fgIO3QTCq%2B2RNzR4IIkOhsKwACoYZhVAQ8b8llEACIwuF0m6/MmYSnIYiYHGYYyciAYrE4piTBH466EzmU/wAdwCcL2a2ZLDEtFQyAgiQAXphUFQIJzJpM2dK0XgwRAwGAFUqsBKGVLCYTYsQMUbVisALKYFgkAE7MLEEhLACsjgY3rNaxdrtRvIIcwYOwCcm83nNhIZjJRaPtcIAtBdsVUVcyy0xswXMIrixd/DSK2mM1m4/nE5hk8RU4WsObc/SbrL0QwlDVBXDjXDKQg%2BehSGPMVWnUjOyyKVT689VTseXyCAKhaua6irTsbWB50x0Gv4wmAPSPnYAdUYqfw8N8fXh9DC6J9BuhI3ugLb9pgZ4IsypRKJKwFos%2BOxCFC86igoOzYAwuxYMQOw/keCGoqOR4sLELagVBhLyggdDwhApGxCWaxYdSBD3kRrqMS2jHMaxNJUWiQ4EgmOxIQAYksjhrEsiKOMA8JYH2mCuIwOwANKwkeqa0QwuquJxvGNmx4F1g6kEdiJqLCRaXY9jpC6Dkyw7XAAbqgeBge6GLGB8fQzluoEcVZm5cjs3H7MylGWbZOw0XRl6McFsVuh6gheisKzXOgvIKHBWAYdg9BsIICggHsKzhrEEZRt6pBJYJqLeelEDeq6ACSDAIGIK4AeZ5WrOG6A1dGmVLkZLGnjFok7M1BAZSshIBFJMnXJIyAIL%2BhXFYwBADZV1WRqNKzjf6THGTSsYPlxZ08WdfFNrS01Cc5IVzQt6aZiNMZOXm7I3O5nk7FQvICn5BABWFQXwSFJFnVB8X0Je0POoZt2RTs0XXSBC4UQuD1sY1wOgwxZ1XRuNmUzcIrqn4EDJTKs6Y7jGOfe212gS2fgTgQU5YBAoFLlw5MhZzGPc2EvNCoLOwrCLsVi3uEuTtLC5Lms8s47eXPjpLfOYALas7JImtonNvl4P5QVQSDmBg5bEPW89dkpjsGi/b8HDTLQnDhrwngcFopCoJwjjgrM8zwncPD1ZoXvTAA1iA4ZcCcEThhEXAp5IXAAJzhuGAAcawRPonCSP7cfB5wvBlRoseB17pBwLASDUpgyCuEQZAUBA1TAAoyiGKUQgIKg8oBzHaBkXQOLJEP/i0KP48B0H0%2BxHR4TIMAXBrI06%2BbwErCLLwB/0MQADyXfLxPVft8g1zEAPNekPflT4AHvD8IIIhiOwUgyEEIoFQ6hG6kF0I0AwRgQCmGMOYD4wQyqQGmKgWI5QyocF4KgVyIZPKYCQfTUgkJBB4DYAAFVQC4Ah0wFARwWHoI4fgF4jzHrfbgvB5TECYLETgPBva%2B0rmAkOHBsCqA7l3EgOxVCFwiEWCIkgdjAGQMgHYu8ThcEvI4JcuBCCSOjpMXge1G6mlIKBbohCk6SELicNYkg1iFysUXDQkgc7yLLhwCupBV5YJfnXBuWgTE%2Bw4CsQRQdhGGLjiYnBxBEh2EkEAA}{Dynamically adding nodes}
\end{aufgabe}

\section*{Anmerkungen}
\begin{enumerate}
    \item Stack vs. Heap Allocation: \url{https://www.geeksforgeeks.org/stack-vs-heap-memory-allocation/}
    \item What is a doubly linked list: \url{https://www.codesdope.com/course/data-structures-doubly-linked-lists/}
    \item Example of a dynamically allocating program: \url{https://github.com/bjoekeldude/doubly-linked-list-minimal}
\end{enumerate}

\end{document}
